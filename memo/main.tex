\documentclass[9pt]{ltjsarticle}
\DeclareSymbolFont{bbold}{U}{bbold}{m}{n}
\DeclareSymbolFontAlphabet{\mathbbold}{bbold}
\newcommand{\bbold}{\mathbbold}
\usepackage{xcolor}
\usepackage{amsmath,amsfonts,amssymb}
\usepackage{enumitem}
\usepackage{ashiato45}
%\usepackage{okumacro}
\def\MARU#1{\textcircled{\scriptsize #1}}
\usepackage{graphicx}
\usepackage{ulem}
\usepackage{framed}
\usepackage{algorithm}
\usepackage{algorithmic}
\usepackage{here}
%\usepackage[twoside]{geometry}
\usepackage{mytheorems}
\usepackage{tikz}
\usepackage{ascmac}
\usetikzlibrary{cd}
\title{Bochnak - Real Algebraic Geometry}
\author{ashiato45 take notes}

\renewcommand{\bf}{\mathbf}


\begin{document}
\maketitle

\section{Ordered Fields, Real Closed Fields}
\label{sec:Ordered Fields, Real Closed Fields}
\subsection{Ordered Fields, Real Fields}
\label{sub:Ordered Fields, Real Fields}
\begin{itemize}
  \item (Definition 1.1.1, ordering of a field):
  $\le$ is an ordering of a field $F$
  $\iff$
  \begin{enumerate}
    \item (total): $\le$ is a total.
    \item (addition): $x\le y \implies x+z \le y+z$
    \item (non-negative and mult.): $0\le x,\, 0\le y\implies 0\le xy$.
  \end{enumerate}
  \item (Small prop.):
  $x\le y,\, z\ge 0 \implies xz\le yz$. \pf
  $x\le y \iff 0\le y-x \iff 0\le (y-x)z \le 0\le yz-xz \iff xz \le yz$.
  \item
  Let's define a ordering of the field of rational function $\R(X)$.
  (Think $X$ as "infinite small").
  \item
  (Example 1.1.2):
  There exists the unique ordering of $\R (X)$ satisfying
  \begin{itemize}
    \item it preserves the ordering of $\R$.\footnote{
    This come from the axiom of fields.
    Or, for $a\in \R$, $0 < X < a$.
    }
    \item $X$ is smaller than any positive real number.
    \item $X$ is positive.
  \end{itemize}
  \pf
  We prove the ordering is unique if any first.
  Let $\FC(f)$ is the coefficient of the lowest term of $f$ for $f \in \R[X]$.
  (FC stands for Following Coefficient twinned with Leading Coefficient)
  Let $\R [X]^+ = \set{f\in \R [X]; \FC()}$
  \begin{enumerate}
    \item
    $0<X$ \pf Requirement.
    \item
    $\Forall{a>0} X<a$ \pf Requirement.
    \item
    $\le$ preserves the ordering of $\R$. \pf Requirement.
    \item
    $\Forall{a>0} \Forall{n\ge 0}X^{n+1} < aX^{n}$
    \item
    $\Forall{a>0}\Forall{m>n\ge 0} X^m < aX^n$
    \item
    $\Forall{a>0,b\in \R}\Forall{m>n\ge 0} bX^m < aX^n$
    \item
    $\Forall{a>0,b\in \R}\Forall{m>n\ge 0} 0<bX^m + aX^n$
    \item
    $\Forall{P(X) \in \R [X]^+} 0 < P(X)$
    \item
    $\Forall{Q(X)\in \R [X]^+} 0 < \frac{1}{Q(X)}$
    \pf
    think of
    \begin{align}
      Q'(X) =
      \begin{cases}
        1/Q^2 ; Q>0
        (-1/Q)^2; Q<0
      \end{cases}.
    \end{align}
    Or assume $0 \ge 1/Q$.  Then multiplying $Q$, $0\ge 1$.
    This contradicts to the axiom of fields.
    \item
    $\Forall{P(X),Q(X)\in \R [X]^+} 0 < \frac{P(X)}{Q(X)}$
    \item
    $\Forall{P(X)\in \R [X]^-,Q(X)\in \R [X]^+} \frac{P(X)}{Q(X)} < 0$
    \item
    \begin{align}
    \Forall{P(X),R(X)\in \R [X], Q(X),S(X)\in \R [X]^+}
    \begin{cases}
      \FC(PS-RQ)>0 &\rightarrow \frac{P}{Q} > \frac{R}{S}\\
      \FC(PS-RQ)=0 &\rightarrow \frac{P}{Q} = \frac{R}{S}\\
      \FC(PS-RQ)<0 & \rightarrow \frac{P}{Q} < \frac{R}{S}
    \end{cases}.
    \end{align}
  \end{enumerate}
  This requirement defines a binary relation $\le$ (check the sign of FC of the numerator\footnote{denominator:分母、numerator:分子}).
  We prove it is exactly an ordering.
  \begin{itemize}
    \item (Reflexivity): Obvious.
    \item (Anti-symmetry): Obvious.
    \item (Total): Obvious.
    \item (Non-negative and mult.):
    Assume $\frac{P}{Q}\ge 0$ and $\frac{R}{S}\ge 0$.
    $\FC(P) \ge 0$ and $\FC(R) \ge 0$ hold.
    Paying attention to managing lowest terms, $\FC(PR)=\FC(P)\FC(R)\ge0$.
    This means $\frac{PR}{QS} \ge 0$.
    \item (Transitivity):
    Assume $\frac{P}{Q}\le \frac{R}{S}$,$\frac{R}{S} \le \frac{T}{U}$ and
    $Q,S,U\in \R [X]^+$.
    By (Non-negative and mult.), they are equivalent to
    $PSU\le RQU$ and $RQU\le TQS$.
    We write for a polynomial $f$ $f$'s $n$-th coeffficient $f_n$.
    For a pair of polynomials $(f,g)$, let $\varphi(f,g)$  is an $n$
    such that $f_0 = g_0,\dots, f_{n-1}=g_{n-1},\, f_n \neq g_n$.
    (If $f=g$, let $\varphi(f,g)=\infty$.)
    $\varphi(PSU,TQS)=\min(\varphi(PSU,RQU), \varphi(RQU,TQS))$ holds.
    Let $N=\varphi(PSU,TQS)$.
    \begin{itemize}
      \item If $N=\infty$ then $\varphi(PSU,RQU)=\varphi(RQU,TQS)=\infty$.
      This means $PSU=RQU=TQS$.
      \item If $N<\infty$ then $(PSU)_0=(RQU)_0=(TQS)_0,\dots,(PSU)_{N-1}=(RQU)_{N-1}=(TQS)_{N-1}$ holds.
      Moreover, $(PSU)_N \le (RQU)_N$ and $(RQU)_N \le (TQS)_N$ hold.
      This means $PSU \le TQS$.
    \end{itemize}
    \item (Addition):
    Obvious.
  \end{itemize}
  \item
  Define $\le$ of $\R (X)$ as
  \begin{enumerate}
    \item
    \begin{align}
      [a_k X^k + \dots + a_n X^n \ge 0 ,\, a_k \neq 0,\, k\le n]
      \iff
      [a_k > 0]
    \end{align}
    \item
    \begin{align}
        [P(X)/Q(X) > 0] \iff [P(X)Q(X)>0]
    \end{align}
  \end{enumerate}
  \item
  This implies immediately
  \begin{align}
    \dots < X^2 < X < 1 < X^{-1} < X^{-2} < \dots .
  \end{align}
  \item (Small prop.): These two rules generates an ordering of a field (Def. 1.1.1).
  \pf TODO.
  \item (Small prop.):
  $\R (X)$ is not archimedean \footnote{Accumulating $1_{\R (X)}$ finitely overwhelms any fixed element of $\R (X)$} i.e.
  \begin{align}
    \Exists{P(X)\in \R (X)} \Forall{n\in \N} n < P(X).
  \end{align}
  \pf Take $P(X)=1/X$.  Fix $n\in \N$. $X<1/n$ holds.
  \begin{align}
    X < \frac{1}{n}
    &\iff
    \frac{1}{n}-X > 0 \\
    &\iff
    \frac{1-nX}{n}>0\\
    &\iff
    1-nX > 0\\
    &\iff
    \frac{1}{X}-n > 0\\
    &\iff
    \frac{1}{X} > n.
  \end{align}
  \item
  This implies $1/X$ is "infinitely large", and
  $X$ is "infinitely small".
  \item
  (Definition, cut):
  (This is probably not the normal definiton...)
  A pair of subsets of $\R$ $(I,J)$ is a cut $\iff$
  \begin{itemize}
    \item $I\cap J = \emptyset$
    \item $I\cup J = \R$
    \item $I < J$ i.e. $\Forall{i\in I}\Forall{j\in J} i < j$.
  \end{itemize}
  \item
  An ordering of $\R (X)$ deterimnes a cut $(I,J)$ where
  \begin{align}
    I=\set{x\in \R; x < X},\, J = \set{x\in \R; X < x}.
  \end{align}
  \begin{align}
    (\text{an ordering of $\R (X)$}) \squig
    (\text{a cut of $\R$})
  \end{align}
  Pay attention to forall $x\in \R$ either $x<X$ or $X<x$ holds because
  the ordering is total.
  \item
  (Definition, $-\infty,a_-,a_+,\infty$):
  Let $a\in \R$.
  $-\infty,a_-,a_+,\infty$ are defined with cuts.
  \begin{itemize}
    \item $-\infty \defeq \paren{\emptyset, \R}$
    \item $a_- \defeq (\openright{-\infty,a},\openright{a,\infty})$
    \item $a_+ \defeq (\openleft{-\infty,a}, \openleft{a,\infty})$
    \item $+\infty \defeq (\R,\emptyset)$
  \end{itemize}
  \item
  (Small prop.):
  $Y=-1/X$ is a bijection between
  $\set{\le (\R (X)); \text{the cut of $\le$ is $-\infty$}}$
  and $\set{\le (\R (Y)) \text{of Def. 1.1.1} }$.
  \footnote{$-\infty$ is the cut defined already.
  Def 1.1.1's cut is $(\openboth{-\infty,0},\openright{0,\infty})$.}

  \pf
  The bijection of $\rightarrow$ part is defining
  a ordering $\R (Y)$ from a fixed ordering $\R (X)$ whose cut is
  $-\infty$.  Define it as
  $P(Y)\ge 0 \iff P(-1/X)\ge 0$.
  We have to check the cut of $P(Y)$ is $(\openleft{-\infty,0},\openboth{0,\infty})$.
  We have to check if $0<Y$ and $Y<(\text{any positive})$.
  The other side is omitted.
  \item
  (Small prop.): $a\in \R$.
  $Y=a-X$ is a bijection between
  $\set{\le (\R (X)); \text{the cut of $\le$ is $a_-$}}$
  and $\set{\le (\R (Y)) \text{of Def. 1.1.1} }$.
  \item
  (Small prop.): $a\in \R$.
  $Y=X-a$ is a bijection between
  $\set{\le (\R (X)); \text{the cut of $\le$ is $a_+$}}$
  and $\set{\le (\R (Y)) \text{of Def. 1.1.1} }$.
  \item
  (Small prop.):
  $Y=1/X$ is a bijection between
  $\set{\le (\R (X)); \text{the cut of $\le$ is $+\infty$}}$
  and $\set{\le (\R (Y)) \text{of Def. 1.1.1} }$.
  \item
  (Small prop.):
  These props states that for each cut, there exists the unique
  ordering.

  \pf At Def. 1.1.1., we have already seen for cut $(\openleft{-\infty,0},\openboth{0,\infty})$
  the ordering whose cut is it is unique.
  These props states that the number of ordering whose cut is
  $-\infty,a_-,a_+,\infty$ equals to
  the number of Def. 1.1.1.'s ordering.
  \item
  (Small prop.):
  This is stated as:
  there exists bijection
  \begin{align}
    \set{\text{all orderings of }\R (X)} \simeq \set{a_+; a\in \R} \cup \set{a_-; a\in \R} \cup \set{-\infty,+\infty}.
  \end{align}
  \item
  (Abuse of term.):
  By the above bijection, we also the orderings by cuts.
  \item
  (TODO, p8):
  Note that the sign of $f\in \R (X)$ for the ordering $a_-$
  is the sign of $f$ on some small open interval $\openboth{a-\epsilon,a}$.
  \item
  (Definition 1.1.3., cone):
  A cone $P$ of a field \footnote{Need not be ordered.} $F$ is a subset $P$ of $F$ such that
  \begin{itemize}
    \item (Addition): $x,y\in P \implies x+y\in P$
    \item (Multiply): $x,y\in P \implies xy \in P$
    \item (Square): $x\in K \implies x^2 \in P$
  \end{itemize}
  The cone $P$ is said to be proper if $-1 \notin P$.
  \item
  (Small example): $\set{0}$ is obviously a proper cone.
  \item
  (Definition 1.1.4., positive cone):
  Let $(F,\le)$ be an ordered field.
  The subset $P=\set{x\in F; x\ge 0}$ is called the positive cone
  of $(F,\le)$.
  \item
  (Proposition 1.1.5., ordering and cone):
  Let $F$ be an ordered field.
  $P$ be a cone.
  \begin{itemize}
    \item
    ($F$ is ordered $(F,\le)$ and $P$ is positive.)
      $\implies$
    ($P\cup (-P) = \R (X)$ and $P$ is proper.)
    \item
    ($P\cup (-P)=\R (X)$ and $P$ is proper.)
    $\implies$
    ($F$ is ordered and its ordering is defined by ($x\le y \iff y-x\in P$))
  \end{itemize}

  \pf
  Prove the first half.
  Proving $-1\ge 0$ is false is sufficient.
  Assume $-1 \ge 0$.
  By (non-negative and mult.), $1 = (-1)\cdot (-1) \le 0$.
  By (addition), adding $+1$ both sides yields $0\le 1$.
  Combining them, $1\le 0 \le 1$.  This means $0=1$.  Contradiction.

  Prove the last half.
  \begin{itemize}
    \item (Reflectivity): Let $x\in F$. Cones always contain $0 = x-x$.  This means $x\le x$.
    \item (Anti-symmetry):
    Let $x,y \in F$ and $x\le y$ and $y\le x$.
    $y-x,x-y\in P$ holds.  Assume $x-y\neq 0$.
    By (Multiply), $-(x-y)^2 = (y-x)(x-y) \in P$.
    Because $x-y \neq 0$, there exists $1/(x-y)\in F$.
    By (Square), $1/(x-y)^2 \in P$.
    $-(x-y)^2 \cdot 1/(x-y)^2 = -1 \in P$.
    This contradicts the properness, so $x-y=0$.
    \item (Transitivity):
    Let $x\le y \in F$ and $y\le z \in F$.
    $y-x \in P$ and $z-y\in P$ hold.
    By (Addition), $z-x = (z-y) + (y-x) \in P$.
    This means $x\le z$.
    \item
    (Total):
    Obvious from $P\cup (-P)= \R (X)$.
    \item
    (Addition):
    Obvious.
    \item
    (Non-negative and Mult.):
    Obvious.
  \end{itemize}
  \item
  (Definition, sum of square):
  The set of sums of squares is denoted by $\sum F^2$.
  \item
  (Small prop.):
  $\sum F^2$ is a cone (not always proper).
  $\sum F^2$ is contained in every cone of $F$ (smallest!).
  \pf Obvious.
  \item
  (Lemma 1.1.7.):
  Let $P$ be a proper cone of $F$.
  \begin{enumerate}[label=(\roman*)]
    \item If $-a \notin P$ then $P[a]=\set{x+ay; x,y\in P}$ is a proper cone of $F$.
    \item There exists an ordering of $F$ and its positive cone $P'$ such that $P \subset P'$.
  \end{enumerate}

  \pf (i)
  Assume that $-1 \in P[a]$. There exists $x,y \in P$ such that
  $-1  = x + ay$.  $(-a)y = x+1$ holds.
  \begin{itemize}
    \item When $y=0$:
    $-1 = x \in P$ holds, but this contradicts that $P$ is proper and $-1 \notin P$.
    \item When $y\neq 0$:
    There exists $1/y \in F$ and $1/y^2 \in P$ by the property of cones.
    \begin{align}
      -a = \frac{x+1}{y} = \ub{y}_{\in P}\cdot \ub{\frac{1}{y^2}}_{\in P \text{(square)}} \cdot (\ub{x}_{\in P}+\ub{1}_{\in P \text{(Square)}}) \in P.
    \end{align}
    This contradicts the assumption.
  \end{itemize}
  Both case lead to contradiction, so $-1 \in P[a]$ is false.
  $-1 \notin P[a]$.

  \pf (ii)
  \begin{enumerate}
    \item $\bbold X$:
    Let
    \begin{align}
      \bbold X = \set{Q' \subset F; P\subset Q',\, \text{$Q'$ is a proper cone}}.
    \end{align}
    \item $Q$:
    $\bbold X$ is not empty because $P \in \bbold X$.
    For a chain of $\bbold X$, its union is a upper bound of it.
    We can apply the Zorn's lemma now, and we obtain a maximal element of $\bbold X$.
    We name it $Q$, $Q$ is a maximal element of $\bbold X$.
    \item
    $Q\cup -Q = F$?
    \begin{enumerate}
      \item $a$: Let $a\in F-Q$.
      \item By (i), $Q[-a]$ is a proper cone.
      \item $Q$ is maximal (by 2), and $Q[-a]$ is a proper cone containing $Q$ (by b).
      Hence $Q=Q[-a]$.
      \item
      Hence $-a \in Q$.
      \item
      (End of a):
      $Q\cup -Q = F$.
    \end{enumerate}
    \item
    $Q$ is proper (by 2) and $Q\cup -Q = F$ (by 3) imply
    (by Prop. 1.1.5.)
    the existence of an ordering $\le$ of $F$.
    And $Q$ is positive in the ordering (by Prop. 1.1.5.).


  \end{enumerate}
  \item
  (Theorem 1.1.8):
  Let $F$ be a field.
  The following properties are equivalent:
  \begin{enumerate}[label=(\roman*)]
    \item $F$ can be ordered.
    \item The field $F$ has a proper cone.
    \item $-1\notin \sum F^2$.
    \item For every $x_1,\dots,x_n \in F$,
    \begin{align}
      \sum_{i=1}^n x_i^2 = 0 \implies
      x_1 = \dots = x_n = 0.
    \end{align}

  \end{enumerate}

  \pf
  \begin{itemize}
    \item (i$\Rightarrow$ ii):
    By Prop. 1.1.5., the positive cone of $F$ is proper.
    So the positive cone satisfies the requirement.
    \item (ii $\Rightarrow$ iii):
    \begin{enumerate}
      \item
      Let the proper cone $P$.
      \item
      By (Small prop.), $\sum F^2$ is the smallest cone, so
      $\sum F^2 \subset P$.
      \item
      Hence
      \begin{align}
        -1 \in F-P \subset F-(\sum F^2).
      \end{align}
      So
      \begin{align}
        -1 \notin \sum F^2.
      \end{align}
    \end{enumerate}
    \item
    (iii $\Rightarrow$ iv):
    \begin{enumerate}
      \item We prove the contraposition.
      Assume $\sum_i x_i^2 = 0$ and $x_1 \neq 0$.
      \item
      $-x_1^2 = \sum_{i=2}^n x_i^2$.
      \item
      Deviding both side by $x_1^2$ (by a, we can divide by $x_1 \neq 0$.)
      \begin{align}
        -1 = \ub{\frac{1}{x_1^2}}_{\in \sum F^2} \ub{\sum_{i=2}^n x_i^2}_{\in \sum F^2} \ub{\in}_{Cone!} \sum F^2.
      \end{align}
    \end{enumerate}
    \item (iv $\Rightarrow $ iii):
    \begin{enumerate}
      \item We prove the contraposition. Assume $-1 \in \sum F^2$.
      \item There exists $a_1,\dots,a_n \in F$ such that $-1 = \sum_{i=1}^n a_i^2$ (by 1).
      \item
      Hence $\sum_{i=1}^n a_i^2 + 1^2 = 0$.
    \end{enumerate}
  \end{itemize}
  \item
  (Definition 1.1.9.):
  A field satisfying (Proposition 1.1.8.) is called real.
  \item
  (Small prop.):
  A real field has characteristic 0.
  \pf
  Assume the characteristic is finite $n$.
  $\sum_{i=1}^n 1^2 = 0$.  This contradicts to (Proposition 1.1.8)'s (iv).
  \item
  (Proposition 1.1.10.):
  \begin{itemize}
    \item $F$: a field such that $\Q \subset P$ (characteristic 0)
    \item $P$: a cone of $F$
  \end{itemize}
  Then
  \begin{align}
     P = \bigcap \ub{\set{Q; [\text{$\le$ is an ordering of $F$}] \land [P\subset Q] \land [\text{$Q$ is a positive cone of $\le$}]}}_{\defeq \bbold X}.
  \end{align}

  \pf
  $\subset $ is obvious.  We prove $\supset$.
  \begin{enumerate}
    \item $a$: Let $a\in F-P$.
    \item $P$ is proper?
    \begin{enumerate}
      \item Assume $-1 \in P$.  (Proof by contradiction)
      \item
      \begin{align}
        a = \ub{\frac{1}{4}}_{\in \sum F^2} [\ub{(1+a)^2}_{\in \sum F^2} \ub{-}_{-1 \in P} \ub{(1-a)^2}_{\sum F^2}] \in \sum F^2 \descsubset{SoS is smallest} P.
      \end{align}
      (the assumption $\Q \subset F$ supports the existence of $1/4$)
      \item
      This contradicts to 1.
    \end{enumerate}
    \item
    $a\notin P$ (by 1), the properness of $P$ (by 2) and (Lemma 1.1.7.)
    show that $P[-a]$ is proper.
    \item
    By (Lemma 1.1.7), there exists an order $\le$ and its positive cone $Q$ such that $P[-a]\subset Q$ (because $P[-a]$ is proper by 3).
    \item
    $a \notin Q$?
    \begin{enumerate}
      \item Assume $a \in Q$. (proof by contradiction)
      \item $-a \in Q$ because $-a \in P[-a] \subset Q$ (by 4).
      \item
      $-a^2 \in Q$ because $Q$ is a cone (by 4), 1 and 2.
      \item
      $a\neq 0$ because $a\notin P$, $P$ is a cone (cones always contain zero).
      \item
      $1/a^2$ is valid and $1/a^2 \in Q$ because $Q$ is a cone.
      \item
      (c) and (e) say
      \begin{align}
        -1 = \ub{-a^2}_{\in Q} \cdot \ub{(1/a^2) }_{\in Q} \in Q.
      \end{align}
      \item
      This contradicts to the properness of $Q$ ((Prop. 1.1.5 ) says the positive cone is proper. )
    \end{enumerate}
    \item
    $P \subset P[-a]\subset Q$.
    \item
    4 and 6 says $Q \in \bbold X$.
    \item
    This shows
    \begin{align}
      a \in F-Q \subset F - (\bigcap \bbold X).
    \end{align}
    \item (End of 1):
    \begin{align}
      F-P \subset F-(\bigcap \bbold X).
    \end{align}
    This means
    \begin{align}
      \bigcap \bbold X \subset P.
    \end{align}

  \end{enumerate}
  \item
  (Corollary 1.1.11.):
  Let $F$ be a field containing $\Q$.
  Then
  \begin{align}
   \sum F^2 = \bigcap {\set{Q; [\text{$\le$ is an ordering of $F$}] \land [\text{$Q$ is a positive cone of $\le$}]}}
  \end{align}

  \pf
  Use (Prop. 1.1.10.) to $\sum F^2$.

\end{itemize}

\subsection{Real Closed Fields}
\label{sub:Real Closed Fields}
\begin{itemize}
  \item (Fact):
  体$F$と、$F$係数既約多項式$f\in F[X]$について、
  $F/(f)$は体になる。
  \item (代数拡大):
  体$F'$が$F$の代数拡大体であるとは、$F'$のすべての元が、
  $F$係数多項式の根になっていること。

  \footnote{戯言:体$F$に、$F$係数既約多項式$f$の根を追加して体にすることができる。
  これは、「$F$にシンボル$X$を追加して、その$X$が$f(X)=0$となる」
  という規則を追加することに外ならないので、
  $F[X]/(f)$は$F$の代数拡大となる。(ただし、拡大したつもりでできていないことはありえる。)}
  \item
  (代数拡大って具体的には?):
  次の命題がある。
  \begin{itemize}
    \item (雪江3.1.23):
    $K$を体、$f$を$K$上既約で$\deg f=n$とする。このとき、次の3つが成り立つ。
    \begin{enumerate}[label=(\arabic*)]
      \item $L=K[x]/(f)$は体で、$[L:K]=n$である。
      \item $\alpha = x+(f)$とおくと、$f(\alpha)=0$
      \item $L$の$K$上の基底として$B=\set{1,\dots,\alpha^{n-1}}$をとれる。
    \end{enumerate}
    \item
    つまり、(1,2)体について既約多項式を考えて、その根が含まれるような
    代数拡大体が存在する。(3)その基底は単項式たち。
  \end{itemize}
  \item (Fact:代数的閉包):
  体$F$について、その代数拡大体で、代数的閉体になっているものが存在し、
  しかも一意である。これを$\ol F$と書くことがある。
  [Yukie, Theorem 3.2.3, Corollary 3.2.4].
  \item (Gaussの対称式の定理):
  See [Cox].
  \item (Definition 1.2.1):
  real field $F$ が real closed field である $\iff$
  $F$ が 非自明な real algebraic extension を持たない
  i.e. $F$の真の代数的拡張 $F_1 \supset F$で、
  \begin{itemize}
    \item $F_1$がreal fieldであり、
    \item $F_1$がalgebraic extensionである
  \end{itemize}
  というようなものは存在しない。
  \item (Theorem 1.2.2.):
  \begin{itemize}
    \item (i$\Rightarrow$ii):
    \begin{enumerate}
      \item (First half starts):
       Let $a\in F$ and $a$ is not a square in $F$.
      \item
      $F[\sqrt{a}] = F[X]/(X^2-a)$.
      Hence $X^2-a$ is (by 1) irreducible,
      $F[X]/(X^2-a)$ is a nontrivial algebraic extension of $F$.
      \item
      (2), (Definition 1.2.1) and (Assumption i) imply
      $F[\sqrt{a}]$ is not real.
      \item
      By (3) and (Theorem 1.1.8, iii), $-1 \in F[\sqrt{a}]$.
      So there exists $x_i,y_i \in F$
      \begin{align}
        -1 = \sum_{i=1}^n (x_i + \sqrt{a}y_i)^2.
      \end{align}
      \item
      Hence $1$ and $\sqrt{a}$ are linearly independent in
      vector space $F[\sqrt{a}]$ \footnote{Remember $F[\sqrt{a}]$ is a quotient of $F[X]$.},
      picking the coefficients of $1$,
      \begin{align}
        -1 = \sum_{i=1}^n x_i^2 + a(\sum_{i=1}^n y_i^2)
      \end{align}
      in $F$.
      \item
      Since $F$ is real and (Theorem 1.1.8, iii)
      \begin{align}
        \ub{-1 - \sum_{i=1}^n x_i^2 }_{\neq 0}= a\sum_{i=1}^n y_i^2.
      \end{align}
      So $\sum_{i=1}^n y_i^2 \neq 0$.
      (Strictly speaking, we need the fact $F$ be an integral domain.)
      \item
      We can divide by $\sum_i y_i^2$,
      \begin{align}
        -a = \frac{1+\sum_{i=1}^n x_i^2}{\sum_{i=1}^n y_i^2} \in \sum F^2.
      \end{align}
      \item
      (End of 1): Forall $a\in F$,
      \begin{itemize}
        \item if $a$ is a square $\rightarrow$ $a \in \sum F^2$,
        \item (by 1-7) if $a$ is not a square $\rightarrow$ $a\in -\sum F^2$.
      \end{itemize}
      Hence
      \begin{align}
        a \in \sum F^2 \cup -\sum F^2.
      \end{align}
      \item
      \begin{align}
        F = \sum F^2 \cup -\sum F^2.
      \end{align}
      \item
      By (Theorem 1.1.8), $\sum F^2$ is a proper cone.
      In this situation,
      (Proposition 1.1.5) says $\sum F^2$ generates
      an ordering of $F$.
      And $\sum F^2$.
      \item
      Assume if another ordering exists.
      Let its positive cone $P$.
      By (Theorem 1.1.5) $P\cup -P = F$.
      $\sum F^2$ is the smallest cone, so
      \begin{align}
        F
        \desceq{9}
        \sum F^2 \cup -\sum F^2
        \subset
        P \cup -P
        =
        F.
      \end{align}
      So $\sum F^2 \cup -\sum F^2 = P \cup -P$.
      Asserting $\sum F^2 \cap -\sum F^2 = \emptyset$ and
      $P \cap -P = \empty$, $\sum F^2 = P$.
      This means the ordering of $P$ and $\sum F^2$ coincides.
      \item  (First half end):
      (10) and (11) says
      there exists unique ordering for $F$ and its positive cone
      is $\sum F^2$.
      \item
      (Last half starts) :
      Let $f\in F[X]$ has odd degree.
      We want to prove $f$ have a root in $F$,
      so we negate this proposition.
      Assume $f$ have no roots in $F$.
      Let $d=\deg f$.
      \item
      We can assume $d > 1$ because
      if $d = 1$ then obviously $f$ have the root in $F$.
      \item
      We can assume that
      polynomials whose degree is $<d$ have a root in $F$.
      \footnote{Strictly, this is proved by the well-ordering set.
      $\set{d; f \text{has no roots}}$ is not empty because the
      assumption.  This have the smallest element.
      Take a polynomial that realize the smallest element.}
      \item (\textbf{ODD})
      $f$ is irreducible.
      \pf Assume $f$ is reducible and there exists decomposition $f=gh$ ($\deg g,\deg h > 0$).
      Then $\deg g,\deg h< d$.
      $\deg g + \deg h = \deg f$ and $\deg f$ is odd, so
      Either $\deg g$ or $\deg h$ is odd.
      Without loss of generality, we can assume $\deg g$ is odd.
      So by (15) $g$ have a root in $F$.
      So $f$ have a root as the root of $g$.
      This contradicts to (13).
      \item
      $F[X]/(f)$ is a nontrivial extension of $F$.
      By (Assumption i), $F[X]/(f)$ is not real.
      So $-1 \bmod (f) \in \sum (F[X]/(f))^2$.
      \item
      There exists $h_i \in F[X], \deg(h_i)<d$ and $g\in F[X]$ such that
      \begin{align}
        -1 = \sum_{i=1}^n h_i^2 + fg.
      \end{align}
      Pay attention to $\deg(h_i)<d \iff \deg(h_i)\le d-1$.
      (the assumption $\deg(h_i)<d$ is from the fact that the ring is a quotient of $f$.)
      \item
      Calculate the degree of both sides of $-1 - \sum_{i=1}^n h_i^2 = fg$.
      \begin{align}
        d + \deg(g)
        &=
        \deg(f) + \deg(g)\\
        &=
        \deg(fg)\\
        &=
        \deg(-1-\sum_{i=1}^n h_i^2)\\
        &\le
        \max_i\deg(h_i^2)\\
        &=
        2\max_i \deg(h_i)\\
        &\le
        2 (d-1)\\
        &=
        2d-2.
      \end{align}
      \item
      $\deg(g)\le d-2$.
      \item
      Seeing the equation of (18), $\deg(-1)=0$ , $\deg(\sum_i h_i^2)$ is odd and
      $\deg(f)$ is odd, so $\deg(g)$ is odd.
      \item
      By (19), (20) and (15), $g$ has a root in $F$.
      Let the root $x$.
      \item
      Substitute $x$ in the equation of (18).
      \begin{align}
        -1 = \sum_{i=1}^n h_i^2(x) + f(x)g(x) \desceq{22} \sum_{i=1}^n h_i^2(x).
      \end{align}
      \item
      This means $-1 \in \sum F^2$.  This contradicts to
      $F$ be the real. (by (12), $\sum F^2$ is a positive cone.)
    \end{enumerate}
    \item
    (ii $\Rightarrow$ iii):
    \begin{enumerate}
      \item (First half starts):
      Let $f\in F[X]$. Set $d = \deg f$.
      We will prove that $f$ have a root in $F[i]$.
      \item
      Write $d=2^m n$ ($n$ is odd).
      \item
      Prove $f$ has a root in $F[i]$ by induction on $m$.
      The case of $m=0$ is obvious from the assumption.
      Assume that the case of $m-1$ holds.
      \item

      Take $y_1,\dots,y_d$ to be the roots of $f$ in $\ol F$.
      \item
      Define for all $h\in \Z$ an element of $F[X]$
      \begin{align}
        g_h = \prod_{1\le \lambda < \mu \le d} (X- y_\lambda - y_\mu -hy_\lambda y_\mu).
      \end{align}
      \item[X6]
      $g_h$ is symmetry in $y_1,\dots,y_d$, so (by Gauss)
      $g_h \in F[(y_1+\dots+y_d),\dots,(y_1\dots y_d)]$.
      \item
      The coefficients of $g_h$ are symmetry in $y_1,\dots,y_d$,
      so (by Gauss) the coefficients of $g_h$ are in $F[(y_1+\dots+y_d),\dots,(y_1\dots y_d)]$.

      \item
      $y_1,\dots,y_d$ are the roots of $f\in F[X]$, so
      $(y_1+\dots+y_d),\dots,(y_1\dots y_d) \in F$.
      \item
      By (5) and (6), $g_h \in F$.
      \item
      \begin{align}
        \deg g_h = \combination{d}{2}
        =
        \frac{d(d-1)}{2}
        =
        \frac{2^m n \cdot (2^m n -1)}{2}
        =
        2^{m-1} \ub{(2^m n -1)n}_{\text{odd}}.
      \end{align}
      \item
      Assumption of induction says $g_h $ have a root in $F[i]$.
      \item
      \begin{align}
        \Forall{h\in \Z} \Exists{1\le \lambda_h < \mu_h \le d} y_{\lambda_h}+y_{\mu_h} + hy_{\lambda_h}y_{\mu_h} \in F[i].
      \end{align}
      \item
      The pairs of $(\lambda_h ,\mu_h)$ is finite, but $h$ runs over $\Z$.
      By pigeonhole principle, there exist different integers $h,h'$ such that
      $(\lambda_h,\mu_h) = (\lambda_{h'},\mu_{h'})$.
      We call this pair $(\lambda,\mu)$.
      \begin{align}
        y_\lambda + y_\mu + hy_\lambda y_\mu,\quad
        y_\lambda + y_\mu + h'y_\lambda y_\mu \in F[i].
      \end{align}
      \item
      \begin{align}
        y_\lambda + y_\mu \in F[i] ,\quad
        y_\lambda y_\mu \in F[i].
      \end{align}
      \item
      2nd degree equation with $F[i]$ coefficients have their roots in $F[i]$?
      \begin{enumerate}
        \item $x^2 = a+bi$ ($a,b\in  F$) have a root in $F[i]$?
        \begin{enumerate}
          \item
          If $b=0$ and $a\ge 0$ \footnote{By assumption ii, we can detrermine if a number is positive or negative.}
          then we can take the square root of $a\in F_+ = \sum F^2$ (assumption ii).
          We call the positive square root of $a\in F_+$ as $\sqrt{a}$.
          If $b=0$ and $a\le 0$ then
          we can take the square root $\sqrt{-a}i$.
          So we can assume $b\neq 0$.
          \item
          Set
          \begin{align}
            L = \sqrt{a^2 + b^2},\quad
            p = \frac{L + (a+bi)}{2},\quad
            M= \frac{\sqrt{(L+a)^2 + b^2}}{2},\,
            q = \frac{p}{M} \sqrt{L}.
          \end{align}
          $M\neq 0$ because $b\neq 0$.
          \item
          $q^2 = a+bi$ holds.
          \pf
          \begin{align}
            q^2
            &=
            \frac{4}{(L+a)^2 + b^2}\cdot \frac{(L+a+bi)^2}{4}\cdot L\\
            &=
            \frac{(L+a)^2 - b^2 + 2(L+a)bi}{L^2 + 2aL + a^2 + b^2}L\\
            &=
            \frac{L^2+2aL+a^2-b^2 + 2(L+a)bi}{2L^2 + 2aL}L\\
            &=
            \frac{2a^2+2aL + 2(L+a)bi}{2L+2a}\\
            &=
            a+bi.
          \end{align}
        \end{enumerate}
        \item
        $ax^2+bx+c = 0$ ($a,b,c\in F[i]$) have a root in $F[i]$.
        \pf
        If $a=0$ then obvious.
        If $a\neq 0$, we can make the completing square, so
        we can solve the equation by (a).
      \end{enumerate}
      \item
      $y_\lambda,y_\mu \in \ol F$ are
      the roots of $X^2 - (y_\lambda + y_\mu)X + y_\lambda y_\mu$.
      This polynomial have $F[i]$ coefficients by (13).
      By (14), the roots are in $F[i]$, so $y_\lambda,y_\mu \in F[i]$.
      \item (First half ends):
      Hence $f$ has a root in $F[i]$.
      \item
      (Last half starts):
      Let $f\in F[i][X]$.
      \item
      $f\ol f \in F[X]$ holds.
      \pf
      Write $f$ as $\sum_j (a_j + ib_j)x^j$.
      \begin{align}
        f \ol f
        &=
        [\sum_j (a_j + ib_j)x^j] \cdot [\sum_k (a_k + ib_k)x^k]\\
        &=
        [\sum_j (a_j+ib_j)(a_j-ib_j)x^{2j}]
        +
        [\sum_{j>k} (a_j+ib_j)(a_k-ib_k)x^{j+k}]
        +
        [\sum_{j<k} (a_j-ib_j)(a_k+ib_k)x^{j+k}]\\
        &=
        \sum_j (a_j^2+b_j^2)x^{2j} + 2\sum_{j>k} (a_j a_k + b_j b_k)x^{j+k}\\
        &\in
        F[X].
      \end{align}
      \item
      By (1-16), $f\ol f$ has a root $x$ in $F[i]$.
      So $x$ is a root of $f$ or a root of $\ol f$
      \footnote{Assume $x$ is not a root of neither.
      $f(x)\neq 0$ and $\ol f(x)\neq 0$.
      So $f(x)\ol f(x)\neq 0$.  But this is a contradiction.}
      .
      If $x$ is a root of $f$, we complete the proof.
      If $x$ is a root of $\ol f$, $\ol x$ is a root of $f$ (Take an allover conjugate).

    \end{enumerate}
    \item
    (iii $\Rightarrow$ i):
    \begin{enumerate}
      \item
      $F$ is real? (We will prove $-1\notin F$ and use Theorem 1.1.8)
      \begin{enumerate}
        \item The solutions of $X^2 = -1$ are only $i, -i$.
        \footnote{"Since $F[i]$ is a field." is nonsense to me.}
        \item
        $i,-i \notin F$, so $-1 \notin F^2$.
        \item
        $F^2 = \sum F^2$? ($\subset$ is obvious.  We will prove only $\supset$.)
        \begin{enumerate}
          \item
          It is sufficient to prove for all $a,b\in F$ there exists $x\in F$ such that $a^2 + b^2 = x^2$.
          \item
          Let $c,d\in F$ as $a+ib = (c+id)^2$.
          Take $c,d$ exists because $F[i]$ is algebraically closed.
          \item
          We can take $x$ as $c^2+d^2$.\pf
          \begin{align}
            x^2
            &=
            (c^2 + d^2)^2\\
            &=
            c^4 + 2c^2d^2 + d^4\\
            &=
            (c^2-d^2)^2 + 4c^2+d^2\\
            &\desceq{ii}
            a^2 + b^2.
          \end{align}
        \end{enumerate}
        \item
        By (b) and (c), $-1 \notin \sum F^2$.
        \item
        By Theorem 1.1.8, $F$ is real.
      \end{enumerate}
      \item
      $F[i]$ is the only nontrivial algebraic extension
      because $F[i]$ is (by assumption iii) algebraically closed.
      (If we intend to add a root $x$ of $f$ to $F$, $x$ \in $F[i]$.)
      \item
      $-1 \in \sum(F[i])^2$ because $i^2 = -1 \in F[i]$.
      \item
      $F[i]$ is not real.
      \item
      By (2) and (4), all the algebraic extensions of $F$ are not real.
      \item
      By (1) nad (5), $F$ is real closed.
    \end{enumerate}
  \end{itemize}
  \item
  (Theorem 1.2.2.):キモだけ。
  \begin{itemize}
    \item (i$\Rightarrow$ii):
    \begin{itemize}
      \item 「hence, $F[\sqrt{a}]$ is not real」:
      真に拡張してしまっているので、「real fieldである」という方がおかしいということになる。
      \item 「only one possible ordering」:

      $\sum F^2$について、$F$はrealなので、
      $-1 \notin \sum F^2$となり、$\sum F^2$はproper coneになっている。
      よって、      Proposition 1.1.5.より、proper coneによって
      orderingが定まってしまう。
      \item
      「it remains to show that, if $f\in F[X]$ has...」:
      奇数次を持つ$f\in F[X]$が$F$に根を持たなかったとする。
      $\deg f=1$だと根を持つにきまっているから$\deg f >1$としてよい。
      さらに、$d=\deg f$として、$d$より小さい奇数次までは
      根を持っていたとしてもよい。

      すると、$f$は既約であるということになる。なぜなら、
      仮に分解できたら、奇数次を分解するのだから分解した因子のほうに
      $d$次より小さい奇数次の多項式が出てきて、それが仮定より根を持つからである。

      \item
      「The polynomial $g_h$ is symmetric in ...」:
      Factとして、対称多項式は、その係数の基本対称式の和と積(つまり多項式)
      として書くことができる。

      さらに、$y_1,\dots,y_d$を根に持つ多項式が$f$であり、
      $f$は$F$係数だったのだから、根と係数の関係から$y_1,\dots,y_d$
      の基本対称式は$\in F$であり、したがって$g_h\in F[X]$である。

      \item (range over $\Z$):
      ハトノスを使う。

      \item (The field $F$ is real...):
      なぜか順序が逆に書いてあるので、$a^2+b^2=(c^2+d^2)^2$まで読めばできる。
      $c,d$は、代数的閉体と仮定したので存在する。

      \item
      (To conclude..):
      $F$の代数的拡張は、$F[X]/(f)$だが、
      $F[i]$は代数的閉体という仮定から、$f$が既約ならそれは
      2次以下であることがわかる(共役を根に持つから。)。
      (cf, $\C$の2次拡大はない。)


    \end{itemize}
  \end{itemize}
  \item
  (Example 1.2.3):
  \begin{itemize}
    \item ($\R$):
    $\C = \R [i]$, and $\C$ is algebraically closed.  Use (iii).
    \item ($\R_{\alg}$):
    \begin{itemize}
      \item (Field):
      Let $a,b\in \R_\alg$.
      $\Q \subset \Q [a,b] \subset \R_{\alg}$ and $\Q [a,b]$ is
      an algebraic extension of $\Q$.
      So $a+b \in \Q [a,b] \subset \R_{\alg}$ and $ab \in \Q [a,b]$.
      If $a \neq 0$ then $a\inv \in \Q [a] \subset \R_\alg$.
      \item (point):
      $\R_\alg$-coefficient polynomial's roots are in $\R_\alg$.
      \pf
      $x$ is a root of $a_n x^n + \dots + a_0 = 0$ ($a_i \in \R_\alg$).
      \begin{align}
        a_0,\dots,a_n \in \Q [a_0,\dots,a_n].
      \end{align}
      Because $a_0,\dots,a_n$ are algebraic over $\Q$,
      $\Q (a_0,\dots,a_n)$ is an algebraic extension of $\Q$.
      So $[\Q (a_0,\dots,a_n): \Q] < \infty$.
      $\Q (a_0,\dots,a_n)(x)$ is an algebraic extension of $\Q (a_0,\dots,a_n)$.
      So $[\Q (a_0,\dots,a_n,x):\Q (a_0,\dots,a_n)]<\infty$.
      By a fact,
      \begin{align}
        [\Q (a_0,\dots,a_n,x):\Q] = [\Q (a_0,\dots,a_n,x):\Q (a_0,\dots,a_n)] [\Q (a_0,\dots,a_n),\Q] < \infty.
      \end{align}
      So $\Q (a_0,\dots,a_n,x)$ is an algebraic extension of $\Q$ (think of $1,x,x^2,\dots$.  We have a linearly dependent.).
      So $x$ is algebraic over $\Q$, then $x \in \R_\alg$.
      \item (unique ordering):
      We will prove $\sum (\R_\alg)^2 = {\R_\alg}_{\ge 0}$.
      If $a \in \R_\alg$ and $a \ge 0$ then $\sqrt{a} \in \R$.
      Because $a$ is a root of $\R_\alg$-coefficient $X^2 -a$.
      So $\sqrt{a} \in \Q [\sqrt{a}] \subset \R_\alg$.
      We have $\sum (\R_\alg)^2 \cup -\sum (\R_\alg)^2 = {\R_\alg}_{\ge 0}\cup {\R_\alg}_{\le 0} = \R_\alg$.
      So induced ordering by $\R$ is the unique ordering of $\R_\alg$.
      \item (odd polynomial):
      $f = a_n x^n + \dots + a_0$ is odd degree ($a_i \in \R_\alg$).
      $f$ have a root in $\R$.  By (point), the root in $\R_\alg$.
      \item (real closed):
      Use (ii).
    \end{itemize}

    \item
    (Puiseux series with real coefficients):
    $\R (X)\puiseux$ is a set of formal series:
    \begin{align}
      \R (X)\puiseux
      =
      \set{\sum_{i=k}^\infty a_i X^{i/q}; k\in \Z, q\in \N-\zeroset,\, a_i \in \R}.
    \end{align}
    $\C (X)\puiseux$ is similiar.
    $\R (X)\puiseux$ is real closed.
    \pf
    It is known that $\C (X)\puiseux$ is algebraically closed.
    $\C (X)\puiseux = \R (X)\puiseux [i]$ because
    \begin{align}
      \sum_{i=k}^\infty (a_i + \sqrt{-1}b_i)X^{i/q}
      =
      \sum_{i=k}^\infty a_i X^{i/q} + \sqrt{-1}\sum_{i=k}^\infty b_i X^{i/q}.
    \end{align}
    \item
    A positive element of $\R (X)\puiseux$ is a Puiseux
    series of the form $\sum_{i=k}^\infty a_i x^{i/q}$ with $a_k>0$.
    \pf
    We need to prove that it is square.
    Think of a square of an element of $\R (X)\puiseux$.
    \begin{align}
      (\sum_{i=k}^\infty b_i X^{i/2q})^2
      &=
      (\sum_{i=k}^\infty b_i X^{i/2q})(\sum_{i=k}^\infty b_i X^{i/2q})\\
      &=
      \sum_{d=2k}^\infty \sum_{i=0}^{d-2k} b_{k+i} b_{(d-2k)-i} X^{d/2q}\\
      &=
      b_k^2 X^{2k/2q} + (b_k b_{k+1} + b_{k+1}b_k)X^{(2k+1)/2q} + \dots.
    \end{align}
    So we can set $b_k = \sqrt{a_k}$ and $b_{k+1},\dots$ recursively.
    If $a_k < 0$, we cannot make such a process.
  \end{itemize}
  \item
  We use the same interval symbols $\closeboth{a,b},\openboth{a,b}$.
  \item
  (Proposition 1.2.4):
  \begin{itemize}
    \item $\R$: real closed field
    \item$f\in R [X]$
    \item $a,b \in R$: $a<b$
    \item $f(a)f(b) < 0$
  \end{itemize}
  then there exists $x \in \openboth{a,b}$ such that $f(x )=0$.

  \pf
  \begin{enumerate}
    \item By (iii) of the (Theorem 1.2.2), the irreducible factors of $f$ are linear or
    have the form of $(X-(c+di))(X-(c-di))=(X-c)^2 + d^2$ for $c,d\in R$.
    \item
    The latters don't yield opposite sign.
    \item
    There exists a linear factor of $f$ who has opposite sign at $a$ and $b$.
    Name it $g(X)= X-x$.  Now, $x$ is a root of $f$.
    $g(a)g(b) < 0$.
    \item
    $g$ is strictly increasing, so $g(a) < 0$ and $g(b) > 0$.
    So $g(a) < g(x) = 0 < g(b)$.
    By increasingness, $a<x<b$.

  \end{enumerate}
  \item
  (Proposition 1.2.5):
  \begin{itemize}
    \item $R$: real closed field
    \item $f\in R[X]$
    \item $a,b \in R$: $a<b$, $f(a)=f(b)=0$
  \end{itemize}
  then $f'$ has a root in $\openboth{a,b}$.

  \pf
  \begin{enumerate}
    \item We can suppose that $a$ and $b$ are two consecutive roots of $f$,
    i.e. $f$ never vanishes in $\openboth{a,b}$.
    (We can replace nearer roots if they are not consecutive.)
    \item
    Factorize $f$ as
    \begin{align}
      f = (X-a)^m (X-b)^n g
    \end{align}
    where $g$ never vanishes in $\openboth{a,b}$.
    \item
    Differentiate $f$ (algebraic derivative)
    \begin{align}
      f'
      &=
      m(X-a)^{m-1}(X-b)^n g + (X-a)^m n(X-b)^{n-1}g + (X-a)^m (X-b)^n g'\\
      &=
      (X-a)^{m-1}(X-b)^{n-1}\ub{[m(X-b)g + n(X-a)g + (X-a)(X-b)g']}_{\defeq g_1}.
    \end{align}
    \item
    $g(a)$ and $g(b)$ have the same signs because (2) and the contraposition of (Proposition 1.2.4).
    \item
    $g_1(a)=m(a-b)g(a)$ and $g_1(b)=n(b-a)g(b)$,
    hence $g_1(a)$ and $g_1(b)$ have opposite signs.
    \item
    By (Proposition 1.2.4), $g_1$ has a root in $\openboth{a,b}$ and
    so does $f'$.

  \end{enumerate}
  \item
  (Corollary 1.2.6):
  \begin{itemize}
    \item $R$: real closed field
    \item $f\in R[X]$
    \item $a,b\in R$: $a<b$
  \end{itemize}
  then there exists $c\in \openboth{a,b}$ such that
  $f(b)-f(a) = (b-a)f'(c)$.
  \pf
  \begin{enumerate}
    \item Let $g(x) = f(x) - [\frac{f(b)-f(a)}{b-a}(x-a)+f(a)]$.
    \item
    $g(a)=0$ obviously holds.
    \begin{align}
      g(b)
      =
      f(b) - [\frac{f(b)-f(a)}{b-a}(b-a)+f(a)] = 0.
    \end{align}
    \item
    Apply (Proposition 1.2.6) to $g$.
  \end{enumerate}
  \item
  (Corollary 1.2.7):
  \begin{itemize}
    \item $R$: real closed field
    \item $f\in R[X]$
    \item $a,b \in R$: $a<b$
    \item $f'$ is positive (resp. negative) on $\openboth{a,b}$
  \end{itemize}
  then $f$ is strictly increasing (resp. strictly decreasing) on $\closeboth{a,b}$.

  \pf
  Obvious from (Corollary 1.2.6).
  \item
  (Definition 1.2.8):
  \begin{itemize}
    \item $R$: real closed field
    \item $f,g \in R[X]$
  \end{itemize}
  The strum sequence of $f$ and $g$ is the sequence of
  polynomials $(f_0,\dots,f_k)$ define as follows
  \begin{itemize}
    \item $f_0 = f$
    \item $f_1 = f'g$
    \item $f_2 = f_1 q_2 - f_0$ with $q_2 \in R[X]$ and $\deg(f_2)  < \deg(f_1)$.
    \item $f_i = f_{i-1} q_i - f_{i-2}$ with $q_i \in R[X]$ and $\deg(f_i) < \deg(f_{i-1})$
    \item $f_k$ is a GCD of $f$ and $f'g$. \footnote{GCD has an ambiguity of unit.}
  \end{itemize}

  \pf
  The Strum sequence is determined by $f$ and $g$ because
  for $i\ge 2$, $f_i$ is determined by division algorithm.
  The stop of the sequence is Euclid algorithm.
  ($\GCD(f_0,f_1)=\GCD(f_1,f_2)=\dots.$)
  \item
  (Definition, sign change):
  Define for an sequence $(a_0,\dots,a_k)$ where $a_0 \neq 0$.
  count one sign change$a_i a_l < 0$
  \begin{itemize}
    \item     with $l=i+1$
    \item $l<i+1$ and $a_j = 0$ ofr every $j (i < j < l)$.
  \end{itemize}

  i.e. the number of successive subsequence such that $ab<0$ and
  \begin{itemize}
    \item $(a,b)$
    \item $(a,0,b)$
    \item $(a,0,0,b)$
    \item $\vdots$
  \end{itemize}

  I write the sign change of a sequence $(a_0,\dots,a_k)$ as $\SC(a_0,\dots,a_k)$
  on my own.
  \item
  (Definition, $v(f,g;a)$):
  \begin{itemize}
    \item $f,g \in R[X]$
    \item $a\in R$: $a$ is not a root of $f$ (To satisfy the hypothesis of the definition of sign change.)
    \item $(f_0,\dots,f_k)$: the Strum sequence of $f$ and $g$
  \end{itemize}
  \item (Theorem 1.2.9, Sylvester's Theorem):
  \begin{itemize}
    \item $R$: real closed field
    \item $f,g \in R[X]$
    \item $a,b \in R$: $a<b$, neither $a$ nor $b$ are roots of $f$
  \end{itemize}
  then
  \begin{align}
    \#\set{x\in \openboth{a,b}; f(x) = 0 \land g(x)>0} -
    \# \set{x\in \openboth{a,b}; f(x)=0 \land g(x) < 0} =
    v(f,g;a) - v(f,g;b).
  \end{align}
  (We don't care of multiplicity.)
  \item
  \begin{enumerate}
    \item Define $(g_\bullet)$ as
    \begin{align}
      (g_0,\dots,g_k) \defeq (f_0/f_k,\dots,f_k/f_k).
    \end{align}
    \item
    Let $x$ is not a root of $f$.
    Because $f_k | f$, $x$ is not a root of $f_k$.
    So division by $f_k(x)$ is reasonable and
    a sequence $(g_0(x),\dots,g_k(x))$ makes sense.
    \item
    The signs of $(f_0(x),\dots,f_k(x))$ and $(g_0(x),\dots,g_k(x))$
    coincide for each $x\in R$.

    (Book: This implies for all $x\in R\setminus \set{\text{roots of $f$}}$\footnote{
    the exclusion of roots is needed for the definition of $\SC(f_0(x),f_1(x))=\SC(f(x),f'g(x))$.
    }
    \begin{align}
      \SC(f_0(x),f_1(x)) = \SC(g_0(x),g_1(x)),\quad
      \SC(f_{i-1}(x),f_i(x),f_{i+1}(x)) = \SC(g_{i-1}(x),g_i(x),g_{i+1}(x)).
    \end{align}
    )
    \item
      \begin{align}
        \set{\text{roots of $g_0$}} = \set{\text{roots of $f$}} \setminus \set{\text{roots of $g$}}
      \end{align}
      ?
      \begin{enumerate}
        \item Calculate $g_0$.
        \item
        Assume
        \begin{align}
          f &= (x-a_1)^{A_1} \dots (x-a_l)^{A_l} (x-b_1)^{B_1}\dots (x-b_m)^{B_m} F(x)\\
          g &= (x-b_1)^{C_1}\dots (x-b_m)^{C_m} (x-c_1)^{D_1}\dots (x-c_n)^{D_n} G(x)
        \end{align}
        where $a_\bullet,b_\bullet,c_\bullet$ are different and
        $F,G$ don't have root in $R$.
        $A_\bullet,B_\bullet,C_\bullet,D_\bullet \ge 1$.
        ($b_\bullet$ are the common roots of $f$ and $g$, $a_\bullet$ are the roots only of $f$, $c_\bullet$ are the roots only of $g$.)
        \item There exists $F_1(x) \in R[x]$ such that
        \begin{align}
          f' =
          (x-a_1)^{A_1-1}\dots (x-a_l)^{A_l-1} (x-b_1)^{B_1-1}(x-b_m)^{B_m-1} F_1(x).
        \end{align}
        where $F_1$ doesn't disappear at $a_\bullet,b_\bullet$ (Calculate!).
        \item
        \begin{align}
          f'g =
          (x-a_1)^{A_1-1}\dots (x-a_l)^{A_l-1} (x-b_1)^{B_1 + C_1 - 1}(x-b_m)^{B_m+C_m-1} F_1(x)G(x).
        \end{align}
        \item
        \begin{align}
          \GCD(f,f'g)
          &=
          (x-a_1)^{A_1-1}\dots (x-a_l)^{A_l-1}(x-b_1)^{B_1}\dots (x-b_m)^{B_m}\\
          &\quad \times \ub{\GCD\left( \ub{(x-a_1)\dots (x-a_l) F_1(x)}_{H_1(x)\defeq}, \ub{(x-b_1)^{C_1-1} \dots (x-b_m)^{C_m-1} F_1(x) G(x)}_{H_2(x)\defeq} \right)}_{ H(x) \defeq}.
        \end{align}
        (by (b), $C_\bullet -1 \ge 0$)
        \item
        By the definition of GCD ($H|H_1$ and $H|H_2$), the roots of $H$ is a root of $H_1$ and $H_2$.
        \item
        If $\xi$ is not a root of $H_1$ or not a root of $H_2$ then
        $\xi$ is not a root of $H$.
        \item
        By (c), $b_\bullet$ are not roots of $H_1$.
        \item
        By (b) and (c), $a_\bullet$ are not roots of $H_2$.
        \item
        (g,h,i) implies $a_\bullet,b_\bullet$ are not roots of $H$.
        \item
        By (e, j),
        \begin{align}
          \frac{f}{\GCD(f,f'g)} = (x-a_1)\dots (x-a_l) \ub{\frac{F(x)}{H(x)}}_{\in R[x]}.
        \end{align}
        because $f/\GCD(f,f'g) \in R[x]$.  By (b), $\frac{F}{H}$ have no root at $a_\bullet$.
        \item
        By definition of $g_0$, $g_0 = \pm f/\GCD(f,f'g)$.
        $a_\bullet$ were the roots of $f$ which are not roots of $g$.
      \end{enumerate}
      \item
      $i\in \set{0,\dots,k}$, $g_{i-1} \perp g_i$ \pf
      Because $f_k = \pm \GCD(f,f'g)$, $\GCD(g_0,g_1)=1$.
      Next $f_i = f_{i-1}q_i - f_{i-2}$, so
      $(f_{i-1},f_{i-2}) = (f_{i-1}, f_{i-1}q_i - f_i) = (f_{i-1},f_i) = (1)$.
      \item
      Let $c$ be a polynomial $g_i$.
      \begin{enumerate}
        \item When $g_i = g_0$. $c$ is a root of $g_0$.
        (Pay attention to Proposition 1.2.4 intermediate-value theorem from here!)
        \begin{enumerate}
          \item By (5), $c$ is not a root of $g_1$.
          (the sign change happens immediately!)
          \item
          By (4), $f(c)=0$ and $g(c)\neq 0$.
          \item
          We define the sign of $f'(c_-)$ as the sign of $f'$ immediately to the left of $c$.
          We can take "immediate left" because
          the roots of $f'$ are finite and intermediate-value theorem.
          We define $f'(c_+)$ similarly.
          \item
          $f'(c_-) \neq 0$ and $f'(c_+) \neq 0$. \pf
          Assume $f'(c_-)=0$.  We have infinitely many "immediate left" points,
          so $f'$ vanishes at infinitely many points. Polynomial $f' \equiv 0$.
          $f(c)=0$ (ii) and $f'\equiv 0$ imply $f\equiv 0$.
          This contradicts to "neither $a$ nor $b$ are roots of $f$".
          \item
          By (ii,iv), we have eight cases:
          \begin{align}
            (g(c),f'(c_-),f'(c_+)) =
            (+++), (++-), (+-+), (+--), (-++), (-+-), (--+), (---).
          \end{align}
          \item
          In every case as $x$ passes through $c$
          \footnote{Immediate left $c_-$ and right $c_+$} ,
          the number of sign changes in $(f_0(x),f_1(x))$
          \begin{itemize}
            \item $g(c)>0 \implies$ decreases by $1$
            \item $g(c)<0 \implies$ increases by $1$
          \end{itemize}
          (We don't have to think of the case of $g(c)=0$ because ii)
        \end{enumerate}
        \item
        \begin{enumerate}
          \item When $i=1,\dots,k$.
          \item
          $g_i(c)=0$
          \item
          By (5), $g_{i-1}\perp g_i$ and $g_i \perp g_{i+1}$.
          This means $g_{i-1}(c)\neq 0$ and $g_{i+1}(c)\neq 0$.
          \item
          $g_{i-1}(c)g_{i+1}(c) < 0$. \pf
          By definition of a sequence, $g_{i+1}=g_i q_{i+1} - g_{i-1}$
          \begin{align}
            g_{i+1}(c) = g_i(c) q_{i+1}(c) - g_{i-1}(c) \desceq{ii} -g_{i-1}(c).
          \end{align}
          \item
          The signs of $(f_{i-1}(x),f_i(x),f_{i+1}(x))$ is
          $(++-),(-++),(+--),(-++)$.
          \item
          The number of sign changes in $(f_{i-1}(x),f_i(x),f_{i+1}(x))$ does not change
          passing $c$.
        \end{enumerate}
      \end{enumerate}
      \item By intermediate-value theorem and (6),
      the sign changes in intervals made by roots of $g_\bullet$.
      We can chase the sign changes only by watching roots of $g_\bullet$,
      and the way the change happens is (a) or (b) (may happen simultaneously).
      \item
      \begin{align}
        \#\set{x\in \openboth{a,b}; f(x) = 0 \land g(x)>0} -
        \# \set{x\in \openboth{a,b}; f(x)=0 \land g(x) < 0} =
        v(f,g;a) - v(f,g;b).
      \end{align}
  \end{enumerate}
  \item
  (Example of sign change):
  \begin{align}
    (+-+-+-+,6)
    &\to
    (--+-+-+, 5)\\
    &\to
    (-++-+-+, 5)\\
    &\to
    (+++-+-+, 4).
  \end{align}
  \item (TODO): Why "real closed"?
  \item (Corollary 1.2.10, Strum's Theorem):
  \begin{itemize}
    \item $R$: real closed field
    \item $f\in R[X]$
    \item $a,b\in R$: $a<b$, $f(a)\neq 0,\, f(b)\neq 0$
  \end{itemize}
  then
  \begin{align}
    \# \set{\text{roots of $f$}} = v(f,1; a) - v(f,1;b).
  \end{align}
  \pf
  Apply 1.2.9 with $g=1$.
  \item
  (Lemma 1.2.11):
  \begin{itemize}
    \item $R$ : real closed field
    \item $f=a_n X^n + \dots + a_0 \in R[X],\, a_n \neq 0$
    \item $M=1+\abs{a_{n-1}/a_n} + \dots + \abs{a_0/a_n}$
  \end{itemize}
  then
  \begin{itemize}
    \item $f$ never vanishes on $\openright{M,+\infty}$ and its sign is the sign of $a_n$.
    \item $f$ never vanishes on $\openleft{-\infty,-M}$ and its sign is the sign of $(-1)^na_n$.
  \end{itemize}
  \item
  \pf We prove the first one.
  \begin{enumerate}
    \item Let $x\in R$, $\abs{x}\ge M$.  (Aim: $f(x)\neq 0$ and $\sign f(x) = \sign a_n$)
    \item
    Triangle ineq. holds.
    \begin{align}
      \abs{\frac{a_{n-1}}{a_n} x\inv + \dots + \frac{a_0}{a_n}x^{-n}}
      \descle{$M>1$}
      (\abs{b_{n-1}}+\dots + \abs{b_0})M^{-1} < 1.
    \end{align}
    \item
    \begin{align}
      -1 < \frac{a_{n-1}}{a_n} x\inv + \dots + \frac{a_0}{a_n}x^{-n} < 1.
    \end{align}
    \item
    \begin{align}
      0 < 1+\frac{a_{n-1}}{a_n} x\inv + \dots + \frac{a_0}{a_n}x^{-n}.
    \end{align}
    \item
    \begin{align}
      f(x) = a_n x^n \ub{(1+\frac{a_{n-1}}{a_n} x\inv + \dots + \frac{a_0}{a_n}x^{-n})}_{>0}.
    \end{align}
  \end{enumerate}
  \item
  (Corollary 1.2.12):
  \begin{itemize}
    \item $R$: real closed field
    \item $f,g \in R[X]$
    \item $(f_0,\dots,f_k)$: the Strum sequence of $f$ and $g$
    \item $v(f,g;+\infty) = \SC (\LC f_0,\dots,\LC f_k)$
    \item $v(f,g;-\infty) = \SC (\LC f_0(-X),\dots,\LC f_k (-X))$
  \end{itemize}
  then
  \begin{align}
    \# \set{x\in R; f(x)=0 \land g(x)>0} - \# \set{x\in R; f(x)=0 \land g(x)<0}
    =
    v(f,g;-\infty) - v(f,g;+\infty).
  \end{align}
  \item
  \pf
  \begin{enumerate}
    \item
    Let $M$ is larger than all the roots of $f_0,\dots,f_k$ are in $\openboth{-M,M}$.
    (This is possible because the roots are finite.)
    \item
    By 1.2.11 (the latter),
    \begin{align}
      v(f,g,+\infty) &= \SC(\LC f_0,\dots,\LC f_k)
      \desceq{1.2.11}
      \SC(f_0(M),\dots,f_k(M))
      =
      v(f,g,M),\\
      v(f,g,-\infty) &=
      \SC(\LC f_0(-X),\dots,\LC f_k(-X))
      =
      \SC((-1)^{\deg f_0} \LC f_0,\dots, (-1)^{\deg f_k} \LC f_k)
      \desceq{1.2.11}
      v(f,g,-M).
    \end{align}
    \item
    By 1.2.9,
    \begin{align}
      \# \set{x\in R; f(x)=0 \land g(x)>0} - \# \set{x\in R; f(x)=0 \land g(x)<0}
      &=
      \# \set{x\in \openboth{-M,M}; f(x)=0 \land g(x)>0} - \# \set{x\in \openboth{-M,M}; f(x)=0 \land g(x)<0}\\
      &=
      v(f,g;-M) - v(f,g;M)\\
      &=
      v(f,g;-\infty) - v(f,g;+\infty).
    \end{align}
  \end{enumerate}
  \item
  (Remark 1.2.13):
  \begin{itemize}
    \item $f\in R[X]$: monic, square free, degree $n$ \footnote{$f\perp f'$, or have no multiple roots}
  \end{itemize}
  then
  $f$ has $n$ roots $\iff$
  the Strum sequence of $f$ and $1$ have $n+1$ length ($(\ub{f_0}_{=f},\ub{f_1}_{=1\cdot f' = f'},\dots,f_n)$) and
  leading coefficients of $f_0,\dots,f_n$ are positive.
  \item
  \pf
  $\Rightarrow$:
  By 1.2.12, $v(f,1;-\infty)-v(f,1;+\infty)=n$.
  Because $\deg f = n$, the length of the Strum sequence is $\le n+1$.
  So $0\le v(f,1;-\infty)\le n$ and $0\le v(f,1;+\infty) \le n$.
  $v(f,1;-\infty)$ must be $n$ and $v(f,1;+\infty)$ must be $0$.
  The signs of $(f_0(+\infty),\dots,f_n(+\infty))$ \footnote{leading coefficients}
  are $(++\dots ++)$ because $f$ is monic.

  $\Leftarrow$:
  By the definition of Strum sequences,
  $\deg f_i = n - i$.  The signs of
  $(f_0(+\infty),\dots,f_n(+\infty)) = (++\dots ++)$.
  These two imply $(f_0(-\infty),\dots,f_n(-\infty)) = (\dots \pm \mp)$.
  \item
  (Proposition 1.2.14, Descartes's Lemma):
  \begin{itemize}
    \item $R$: real closed field
    \item $f=a_n X^n + \dots + a_k X^k \in R[X]$ with $a_n a_k \neq 0$
  \end{itemize}
  then
  \begin{align}
    \# \set{x \in \openboth{0,+\infty}; f(x)=0} \le \SC(a_n,\dots,a_k).
  \end{align}
  \item
  \pf
  \begin{enumerate}
    \item
    Think of $n=1$.
    \begin{enumerate}
      \item  $f$ has the form of
      $f=a_1X + a_0$ or $f=a_1X$.
      \item If $f=a_1 + a_0$,
      \begin{itemize}
        \item $a_1 > 0$ and $a_0 < 0$:$f$ has one positive root.  SC is $1$.
        \item $a_1 >0 $ and $a_0 > 0$ : $f$ has no positive roots.  SC is $0$.
        \item $a_1 < 0$ and $a_0 < 0$ : $f$ has no positive roots.  SC is $0$.
        \item $a_1 < 0$ and $a_0 > 0$ : $f$ has one positive root.  SC is $1$.
      \end{itemize}
      OK.
      \item
      If $f=a_1 X$,
      $f$ has no positive roots (it is zero!) and
      SC is $0$.  OK.
    \end{enumerate}
    \item
    So if $n=1$, OK.
    \item We prove the statement by induction.  The base case is already proved in (1-2).
    We assume the case of $n-1$.
    \item
    We can assume $X$ does not divide $f$, i.e. $a_0 \neq 0$,
    because we can divide $X$ as many as possible.
    The division doesn't change the SC nor positive roots.
    So $f=a_n X^n + \dots + a_q X^q + a_0$ where
    $a_n \neq 0,\, a_q \neq 0,\, a_0 \neq 0$.
    \item
    $f'=na_n X^{n-1} + \dots + qa_q X^{q-1}$.
    \item
    We can apply the hypothesis of induction,
    \begin{align}
      \#\set{x\in \openboth{0,+\infty}; f'(x)=0} \le \SC(a_n,\dots, a_q).
    \end{align}
    \item
    Let $c \in R$ be the smallest positive root of $f'$.
    If it does not exist, let $c = +\infty$.
    \item
    By intervalue theorem,
    \begin{align}
      \sign a_q = \ub{ \sign \openboth{0,c}}_{\text{intervalue}}
    \end{align}
    \item
    $f(0) = a_0$.
    \item
    \begin{itemize}
      \item The case $f$ has a root in $\openboth{0,c}$:
        \begin{enumerate}
          \item Seeing the variation of $f$,
          $a_q a_0 < 0$ is necessary for the case.
          \item
          \begin{align}
            \SC(a_n,\dots,a_q) + 1 = \SC(a_n,\dots,a_q,a_0).
          \end{align}
          \item
          By Rolle's theorem, there is exactly one root in $\openboth{0,c}$.
          \pf If any, the property of $c$ in (7) is wrong.
          \item
          So by intervalue theorem
          \begin{align}
            \# \set{\text{positive roots of $f$}} -1 \le \# \set{\text{positive roots of $f'$}}.
          \end{align}
          (for a interval of $f$'s roots, there exist at least one root of $f'$ \footnote{Assume $f$ is not zero.})
          \item
          \begin{align}
            \# \set{\text{positive roots of $f$}}
            &\descle{d}
            \# \set{\text{positive roots of $f'$}} + 1\\
            &\descle{6}
            \SC(a_n,\dots,a_q) + 1\\
            &\desceq{b}
            (\SC(a_n,\dots,a_0)-1) + 1\\
            &=
            \SC(a_n,\dots,a_0).
          \end{align}
        \end{enumerate}
      \item Otherwise:
        \begin{enumerate}
          \item
          By assumption, there are no roots in $\openboth{0,c}$.
          \item
          So (similar to 10-d)
          \begin{align}
            \# \set{\text{positive roots of $f$}} \le
            \# \set{\text{positive roots of $f'$}}.
          \end{align}
          \item
          \begin{align}
            \# \set{\text{positive roots of $f$}}
            &\descle{b}
            \# \set{\text{positive roots of $f'$}}\\
            &\descle{6}
            \SC(a_n,\dots,a_q)\\
            &\le
            \SC(a_n,\dots,a_0).
          \end{align}
        \end{enumerate}
    \end{itemize}
    \item
    In both cases of (10),
    \begin{align}
      \# \set{\text{positive roots of $f$}} \le \SC(a_n,\dots,a_0).
    \end{align}
  \end{enumerate}

\end{itemize}

\subsection{Real Closure of an Ordered Field}
\label{sub:Real Closure of an Ordered Field}
\begin{itemize}
  \item (Definition 1.3.1):

  \item (Lemma 1.3.3):
  \begin{itemize}
    \item $(F,\le)$: ordered field
    \item $R$: algebraic extension of $F$
    \item $R'$: real closed field containing $F$ and preserving the ordering of $F$
    ($F \hookrightarrow R'$)
    \item $L$: intermediate field between $F$ and $R$.
    ($F \subset L \subset R$, not usually order preserving)
    \item $L_1$: extension of finite degree of $L$
    ($F \subset L \subset L_1 \subset R$)
    \item $\varPhi \colon L\to R'$
  \end{itemize}
\end{itemize}



\end{document}
